In this section we will describe the different algorithms and memory
layouts we have used.

\subsection{Simple multiplication}

\subsubsection{Row-based layout}
We started out implementing the simple conventional $O(n^3)$ matrix
multiplication algorithm where we stored the matrices using a row
based layout.

\todo{Expectations of cache faults here....}

We do not expect any significant amount of branch mispredictions.

\subsubsection{Combined row-based and column-based layout}

In order to improve the number of cache faults, we have tried to use a
column base layout in the right operand in the multiplication. We
expect this to give us a bit better cache performance. This approach
has the drawback of limiting a matrix only to be used on one side of a
multiplication.

\subsection{Recursive multiplication}

...

\subsubsection{Z-curve layout}

...

\todo{* Specielle instruktioner til at parrelelisere simple multiplikationer}
\todo{* Flertr�det}
