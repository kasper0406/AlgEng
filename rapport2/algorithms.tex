In this section we will describe the different algorithms and memory
layouts we have used.

\subsection{Simple multiplication}

\subsubsection{Row-based layout}
We started out implementing the simple conventional $O(n^3)$ matrix
multiplication algorithm where we stored the matrices using a row
based layout.

\todo{Expectations of cache faults here....}

We do not expect any significant amount of branch mispredictions.

\subsubsection{Combined row-based and column-based layout}

In order to improve the number of cache faults, we have tried to use a
column base layout in the right operand in the multiplication. We
expect this to give us a bit better cache performance. This approach
has the drawback of limiting a matrix only to be used on one side of a
multiplication.

\subsection{Recursive multiplication}

For exploiting more kinds of layouts we implemented the recursive algorithm.

\subsubsection{Z-curve layout}

The first idea was to use a Z-curved layout. One major advantage of this layout is that improves locality on all levels of the recursive multiplication. The drawback is that the index calculations are time consuming. On x86 we get approximately 50 bitwise operations each time. However, this can be improved by incrementally constructing the Z-curve numbers or by precomputing offsets at base cases. The upcoming Intel Haswell architecture has support for bit permutations which will improve the situation.

In the recursive algorithm we used 8*8 as a base case for switching to the naive algorithm.

\todo{Lav dem inkrementielt?}

\subsubsection{Tiled layout}

A tiled layout is a compromise between locality and performance of index calculations. When the recursive algorithm reaches blocks of the same size as the tiles, it switches to the naive algorithm. And the naive algorithm functions without any modifications.

\subsection{Special instructions}

...

\subsection{Parallelization}

\subsubsection{Combined row-based and column-based layout}

The naive algorithm which is used for this combined is easily parallelizable because we can assign intervals of rows for each thread. The result matrix is stored in a row-based layout so writing is separated so cache thrashing should not be a problem.

...

\todo{todo, mere l2 cache}