\begin{figure}[h!]
  \centering
  \includegraphics[width=\textwidth]{"../project2/gnuplots/best_single"}
  \caption{Summarised single-core performance.}
  \label{fig:best_single}
\end{figure}

\begin{figure}[h!]
  \centering
  \includegraphics[width=\textwidth]{"../project2/gnuplots/best_parallel"}
  \caption{Summarised multi-core performance on 4 cores with Hyper-Threading.}
  \label{fig:best_parallel}
\end{figure}

In this project we have successfully implemented and tested a range of algorithms and tried different memory layouts and optimizations, for matrix multiplication. Our results shows that memory layouts plays a significant role in reducing running time. We have found that the iterative algorithm with tiled layout and Strassen with a tiled layout in a Z-curve fashion, are the best performing algorithms. Strassen performs a bit better than the iterative but on the other hand, it does not scale as well considering the percentage of parallelized code (figure \ref{fig:amdahl}) and its memory usage. Furthermore, Strassen suffers from numerical instability which might be a problem for some applications.

With the right combination of algorithm, layout and optimizations (SIMD, loop unrolling, stack-like allocation) we have achieved a speedup of \textasciitilde 50 for matrices of size 2048x2048 on a single core setup (figure \ref{fig:best_single}). For 4 cores with HT we achieved a speedup of \textasciitilde 30 (figure \ref{fig:best_parallel}).